\Lecture{Jayalal Sarma, Alexander Shrestov }         % Names of people
           {January 15, 2013}                        % lecture date
           {1}                                       % lecture number
           {Instructions for Preparing Scribe Notes} % lecture title

\noindent
Using the provided \verb|\makeheader| command, 
customize the above header with your name,
lecture date, lecture number, and lecture title. For
example, the above header was generated by typing 
\verb|\makeheader{Ima Student}{January 9, 2012}{15}|{\tt 
\{Instructions for Preparing Scribe Notes\}}.  Your
scribe notes should start with a high-level description
of the lecture, its goal and techniques, and how it
fits in the broader context of the course. In
particular, explain the relationship to the previous
lecture if appropriate; feel free to refer to the
material in previous scribe notes using theorem and
page numbers.  This high-level description should be
two or three solid paragraphs in length.

\section{Organization}
Lecture proper should be presented in a sequence of
sections. For example, you might choose to present
preparatory work in one section, the main results in
another section, and any generalizations or
conclusions in a third section. Do \emph{not} use
any subdivisions within sections (subsections,
subsubsections, etc.).

\section{Some Do's}
We all know from experience that a picture is worth a
thousand words, so be generous with figures. Please be
sure to include all the figures and drawings 
from my lecture, and feel free to include
your own. See Figure~\ref{fig:triangle-circle} for an
example usage of the figure environment. Write in
complete sentences. Be sure to include all
bibliographic references, like so~\cite{textbook}.  The
bibliography must be incorporated using BibTex.  
As with any writing, make sure to spell check your
scribe notes. When finished, please send me the
following files by email: your \LaTeX\ source file
({\tt .tex}), your bibliography file ({\tt .bib}) if
you used one, any figures (ideally in {\tt .pdf}
format), and the resulting typeset document ({\tt
.pdf}).  Remember, preparing scribe notes is a valuable
learning experience and one that allows you to
internalize the material at a new level. Take pride in
your work. 

\begin{figure}
\begin{center}
\includegraphics[width=0.4\textwidth]{triangle-circle}
\end{center}
\caption{A triangle and a circle.}
\label{fig:triangle-circle}
\end{figure}

\section{Some Don'ts}
You must not change the format of the scribe notes in
any way, including font type, font size, pagination,
section numbering, margins, or bibliography style. No
content should spill over into the margins. You
must not use any \LaTeX\ packages that do not come with
the standard \LaTeX\ installation in our department;
as a matter of fact, you should not need to include any
\LaTeX\ packages in addition to those already included
in the template file.


\section{Mathematical Environments}

For your convenience, the scribe note style file comes
with the following mathematical environments
predefined: theorem, lemma, corollary, proposition,
fact, claim, definition, example, assumption, remark,
conjecture, open problem, problem. The environments are
illustrated below.  Please limit yourself to these
environments alone.  

\begin{theorem}
Statement here 
\end{theorem}

\begin{lemma}
Statement here
\end{lemma}

\begin{corollary}
Statement here
\end{corollary}

\begin{proposition}
Statement here
\end{proposition}

\begin{fact}
Statement here
\end{fact}

\begin{claim}
Statement here
\end{claim}

\begin{definition}
Statement here
\end{definition}

\begin{example}
Statement here
\end{example}

\begin{assumption}
Statement here
\end{assumption}

\begin{remark}
Statement here
\end{remark}

\begin{conjecture}
Statement here
\end{conjecture}

\begin{openproblem}
Statement here
\end{openproblem}

\begin{problem}
Statement here
\end{problem}


\noindent
Note that \LaTeX\ automatically numbers these
environments within the lecture number (\thelecture\ in
this case).  The same applies to the numbering of pages
(this page being page \thepage), figures
(Figure~\ref{fig:triangle-circle} above), and
equations:
\begin{align}
a = a_1+a_2+\cdots+a_n.
\end{align}
\noindent
For proofs, use the provided {\tt proof} environment,
illustrated below.

\begin{proof}
Proof goes here.
\end{proof}
